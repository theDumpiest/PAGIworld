\section{Conclusion and Future Work}

The release of PAGI World is accompanied with a call to all AGI and human-level-AI researchers to finally examine the strengths and limits of their preferred approaches. PAGI World allows for researchers to very easily create tasks and microworlds in a 2D world with realistic physics, with no knowledge in how to program. PAGI World can interact with AI agents that are written in virtually any programming language, and the simulation can be run on any major operating system. We have very carefully designed PAGI World to have an extremely low technical barrier, so that many researchers can find common ground upon which to compare their different approaches.

The future of PAGI World is bright. We already have several AI systems in progress whose goals are to solve already-finished PAGI World tasks, and as development continues we hope to greatly increase the number of tasks which are available and the sophistication of the agents which solve those tasks. The library of future tasks, we hope, will diversify and reflect the broad spectrum of tasks which require human-like intelligence. 

One interesting and possibly fertile source of PAGI World tasks is the area of morality. Figure \ref{FIGURE_MORALITY} depicts an example task in which two food objects---an apple and a piece of bacon---are falling down a series of ramps where they will eventually fall off the screen and become unreachable, unless the agent chooses exactly one of them (he will not be able to get both in time). Although not many would consider the choice between apples and bacon to be a moral decision, it is easy to see how such a scenario can be adapted to capture miniature moral dilemmas. For example, if the simulation begins with the agent having knowledge that an apple will save the life of person \textbf{A}, while the bacon will save person \textbf{B} but leave \textbf{A} to die, suddenly Figure (FIG) becomes a moral decision which the agent must make in real time. Examples like these illustrate the wide variety of tasks and demonstrations that can be created with PAGI World.

Special thanks to Kainoa Eastlack for helping us with an earlier version of PAGI World.

This work was supported in part by 

